%!TEX TS-program = pdflatex
%!TEX root = tesi.tex
%!TEX encoding = UTF-8 Unicode


       %%%%%%%%%%%%%%%%%%%%%%
       %                    %
       %  Introduzione.tex  %
       %                    %
       %%%%%%%%%%%%%%%%%%%%%%

\chapter{Introduzione}

Il \TeX{} fu pensato ai lontani tempi in cui i terminali
grafici erano un lusso stravagante e l'unico modo di
comunicare coi calcolatori era di battere tasti della
tastiera. Ancora oggi chi scrive in \TeX{} deve inserire i
comandi di formattazione mescolati al testo ed è meglio
se ha un manuale a portata di mano. Per rialzare
il morale basta che si rammenti che il \TeX{} è ancora
solidamente il più sofisticato sistema di impaginazione
per testi scientifici, e che è gratis.

\section{Prima sezione}

Per cominciare a scrivere in \TeX{} bisogna individuare i
\emph{caratteri speciali}, o \emph{caratteri di
controllo}\index{caratteri di controllo}, che servono a
distinguere il testo dai comandi. Chi lavora con la
tastiera italiana\index{tastiera italiana} dovrà faticare
all'inizio, perché alcuni di quei caratteri importanti
si raggiungono solo attraverso certe combinazioni di
tasti, che per di più cambiano da un sistema
operativo\index{sistema operativo} a un altro. Consultate
il manuale del vostro calcolatore o chiedete aiuto agli
esperti se non vi raccapezzate.

Questa è un'equazione numerata:
\begin{equation}\label{prodottoNotevole}
  (a-b)(a+b)=a^2-b^2.
\end{equation}
Notate il numero di equazione al margine destro. Ne riparleremo nel prossimo capitolo. Notate anche il numero di sezione. In questo esempio di tesi, le sezioni dell'introduzione sono numerate con un singolo numero, mentre nei capitoli seguenti\dots\ beh, guardate voi stessi.

\begin{itemize}

\item Il carattere speciale principe del \TeX{} è il
\verb!\!, chiamato \emph{``backslash''}
\index{backslash (\verb?\?)}
(che significa grosso modo ``fendente inverso'', da non
confondere con~\verb!/!, che è lo ``slash'', o linea di
frazione), che serve per annunciare i comandi. Vi conviene
imparare subito il suo posto sulla tastiera. Come carattere
coincide con il segno meno della teoria degli insiemi
($A\setminus B$)\index{sottrazione insiemistica}, che
però si raccomanda di ricavare col comando
\verb!\setminus!.

\item Le \emph{parentesi
graffe}~\verb!{}!\index{parentesi graffe} sono pure
fondamentali e vanno individuate subito sulla tastiera.
Servono per delimitare le zone di azione di un comando.
Quando le graffe vanno stampate, come nella formula
$\{a\}$, basta farle precedere dal backslash:
\verb!\{a\}!.

\item Il segno del \emph{dollaro}~\$ \index{dollaro,
simbolo del} serve per delimitare le formule inglobate nel
testo. Ad esempio per ottenere $A\setminus B$ scriviamo
\verb!$A\setminus B$!.

\item Gli \emph{accenti}\index{accenti} \verb!`! e \verb!'!
preceduti da backslash servono per accentare il carattere
che segue: per ottenere à, è, é, ù si batte
\verb!à!, \verb!è!, \verb!é!, \verb!ù!. Addirittura
la~ì accentata è patologicamente complicata: si
ottiene battendo \verb!ì!. Se trovate farraginoso
questo metodo di scrivere le lettere accentate, non siete
gli unici. Per questo viene in aiuto il pacchetto
``\verb!inputenc!'', che viene caricato col comando \begin{center}\verb!\usepackage[latin1]{inputenc}!\end{center} (su windows o unix)\index{accenti su windows o unix}.
Con questo pacchetto attivo il \LaTeX{} interpreta
correttamente gli accenti da tastiera ‡, Ë, È, ecc.
Il backslash rimane per gli accenti e segni diacritici
\index{accenti stranieri}
non italiani: \verb!ô! per l'Hôpital,
\verb!ö! per Gödel, \verb!ñ!
per España, \verb!\c{c}! per gar\c{c}on, ecc.
Battendo due volte gli accenti, come in
\verb!``virgolette''! si ottengono le
``virgolette''\index{virgolette} aperte e chiuse.

\item Il \emph{percento}~\verb!%!\index{percento, segno
del} dice al \TeX{} di ignorare il resto della riga di
input, per inserire ``commenti'' al testo sorgente che non
andranno stampati.

\item Il \emph{``cappuccio''}~\verb!^!
\index{cappuccio (\verb?^?)} e
la \emph{sottolineatura}~\verb!_!\index{sottolineatura},
che servono soprattutto per gli esponenti e gli indici
nelle formule: per avere
$a_i^2$ battere \verb!$a_i^2$!, o, indifferentemente
\verb!$a^2_i$!.

\item Le \emph{parentesi
quadre}~\verb![]!\index{parentesi quadre} e i segni di
\emph{disuguaglianza}~$<>$ servono nelle formule
matematiche e in certe tastiere sono nascosti.

\item L'\emph{``ampersand''}\index{ampersand (\verb?&?)}, o
``e~commerciale''~\verb!&!, che separa gli argomenti di un
allineamento.

\item La
\emph{``tilde''}~\verb!~!\index{tilde (\verb?~?)},
che posta fra due parole vieta al \TeX{} di andare a capo
in quel punto. Per esempio scriveremo
\verb!Prof.~Tizio! se siamo pignoli e non vogliamo che
``Prof.'' càpiti a fine riga e ``Tizio'' sia spedito a
capo. L'accento tilde sulla lettera seguente (come in
España o São Paulo) si ottiene con~\verb!\~!
(\verb!España!,
\verb!São Paulo!). La tilde nelle formule matematiche
è diversa: per avere $\tilde a$ bisogna battere
\verb!$\tilde a$!.

\item Lo
\emph{``hash''}~\verb!#!\index{hash (\verb?#?)}
serve come segnaposto per gli argomenti delle funzioni.
Troppo complicato da spiegare qui.

\end{itemize}