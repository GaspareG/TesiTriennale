%!TEX TS-program = pdflatex
%!TEX root = tesi.tex
%!TEX encoding = UTF-8 Unicode

\chapter{Come si fanno le appendici}
  
    \index{appendici}
    
    Le appendici si fanno con \verb!\appendix! seguito da
    \verb!\chapter{...}!

%%%%%%%%%%%%%%%%%%%%%%

\chapter{Esempi di Citazioni Bibliografiche}
  
    \index{bibliografia}
    \index{citazioni}
    
    P\^{y}r{\l}å in~\cite{pyrl} ha poi
    generalizzato i risultati di
    Bi\v{s}ker~\cite{bisker1}.
    
    Il pacchetto \verb!uniudtesi! carica
    automaticamente \verb!hyperref!\index{ipertesto},
    che a sua volta rende ``cliccabili'' i riferimenti 
    bibliografici nel documento elettronico.

%%%%%%%%%%%%%%%%%%%%%%

\chapter{Ambiente GNU/Linux (ad esempio Ubuntu)}

    \index{Linux}
    
    \begin{flushright}Contributo di\\ Leonardo Taglialegne
    \end{flushright}
    
    Gli ambienti GNU/Linux contengono parecchi strumenti utili per
    la stesura di una tesi di laurea, in particolare segnaliamo:
    \begin{itemize}
     \item Kile
     \item KBibTeX
    \end{itemize}
    Il primo è un editor per il \LaTeX, che include una tabella
    dei simboli, la visualizzazione della struttura, evidenziazione
    del codice e simili comodità, e nelle ultime versioni fornisce
    una visualizzazione in anteprima dei risultati di compilazione.
    
    Il secondo è uno strumento di ricerca, modifica ed inserimento
    di citazioni in formato BibTeX.
    
    I pacchetti relativi (ed altri utili) si installano,
    su ambienti Debian e Ubuntu con:
    \texttt{sudo apt-get install kile kile-l10n kbibtex
           texlive-science \\
           texlive-math-extra texlive-lang-italian }
