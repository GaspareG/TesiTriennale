%!TEX TS-program = pdflatex
%!TEX root = tesi.tex
%!TEX encoding = UTF-8 Unicode


     %%%%%%%%%%%%%%%%%%%%
     %                  %
     %  capitolo1.tex   %
     %                  %
     %%%%%%%%%%%%%%%%%%%%

\chapter{Basic tools}

In this chapter we first give a definition of subgraphs similarity 

% In this chapter we first recall some definition on documents similarity and then introduce two different definitions of subgraphs similarity.

%Following we use square brackets [ ] to distinguish multisets from sets, leaving the curly brackets \{ \} to sets 


\section{Similarity indices}

\begin{definizione}\label{def:jaccard}
    Given two set $A$ and $B$ we define the \textbf{Jaccard index} as the ratio between the size of intersection and of the union between the two sets:
    
    \begin{equation}
    J(A,B) = \frac{|A \cap B|}{|A \cup B|}
    \end{equation}
    
\end{definizione}

\begin{definizione}\label{def:bray}
    Given two set $A$ and $B$ we define the \textbf{Bray-Curtis index} as:
    
    \begin{equation}
    BC(A,B) = \frac{2 \times |A \cap B|}{|A| + |B|}
    \end{equation}
    
\end{definizione}

We can easily extended the two previous definition to multiset:

\begin{definizione}\label{def:wjaccard}
    Given two multiset $A = (a_{1}, \ldots, a_{n}) $ and $B = (b_{1}, \ldots, b_{n})$ we define the \textbf{weighted Frequency Jaccard index} as:
    
    \begin{equation}
    FJ(A,B) = \frac{\sum\limits_{i=1}^n { min(a_{i}, b_{i}) } }{\sum\limits_{i=1}^n { max(a_{i}, b_{i}) }}
    \end{equation}
    
\end{definizione}

\begin{definizione}\label{def:wbray}
    Given two multiset $A = (a_{1}, \ldots, a_{n}) $ and $B = (b_{1}, \ldots, b_{n})$ we define the \textbf{Bray-Curtis index} on multiset as:
    
    \begin{equation}
    BC(A,B) = \frac{ 2 \times \sum\limits_{i=1}^n { min(a_{i}, b_{i}) } }{\sum\limits_{i=1}^n {a_{i} + b_{i}}}
    \end{equation}
    
\end{definizione}


\section{Documents similarity}

...

\section{Graphs similarity}

\section{Subgraphs similarity}

...

\section{Sketches}

\subsection{min-wise permutation}
\subsection{bottom-k sketches}

\section{Color Coding}
