%!TEX TS-program = pdflatex
%!TEX root = tesi.tex
%!TEX encoding = UTF-8 Unicode

\chapter{Introduction}

With the spread of Internet and 

\section{Basic definitions}


\begin{definizione}\label{def:graph}
    A \index{graph} is a pair of sets $G=(V,E)$, where $V$ is the set of vertices and $E \subset V \times V$ is the set of edges.
\end{definizione}

If two vertices $u, v \in V$ are connected by an edge they are called extreme of the edge, in this case we denote the edge with the pair $(u, v) \in E$

If $(u,v) \in E \Leftrightarrow (v,u) \in E$ the graph is called undirected, where not specified we will only deal with undirected graphs.

We denote by $N(u) = \{ v : (u,v) \in E \}$ the set of neighbors of the vertex $u$, the cardinality of this set is called degree of $u$.


A sequency of nodes  $v_{1}, v_{2}, \ldots, v_{k}$ is called path if $(v_{i}, v_{i+1}) \in E$ $\forall i = 1, \ldots k-1$; a path is called simple if $v_{i} \neq v_{j}$ $\forall i,j$ $1 \leq i \le j \leq k$.

\begin{definizione}\label{def:subgraph}
    A graph $G' = (V', E')$ is called \index{subgraph} of $G=(V,E)$ if $V' \subset V$ and $E' \subset E$. A subgraph is called induced if $E' = (V' \times V') \cap E$.
\end{definizione}

Note that an induced subgraph $G' = (V', E')$ can be uniquely identified by the set of its vertex $V'$.


\section{The problem}

Given an undirected labeled graph $G=(V,E,L)$ over an alphabet $\Sigma$, an integer $q$
and two  $G1, G2 \subset G$ we want to 


\section{Pratical applications}

%
%
%       %%%%%%%%%%%%%%%%%%%%%%
%       %                    %
%       %  Introduzione.tex  %
%       %                    %
%       %%%%%%%%%%%%%%%%%%%%%%
%
%\chapter{Introduzione}
%
%\section{Definizioni di base e notazione}

%\begin{definizione}\label{def:grafo}
	%Si dice grafo una coppia ordinata $(V, E)$, con $V$ insieme dei vertici e $E \subseteq V %times V$ insieme degli archi.
%\end{definizione}
%	
%	Se due vertici $u, v \in V$ sono congiunti da un arco si dicono estremi dell'arco, in questo caso indichiamo l'arco con la coppia $(u, v) \in E$.
%	
%	Se $(u,v) \in E \Leftrightarrow (v,u) \in E$ il grafo si dice non orientato. Considereremo solo grafi non orientati se non specificato altrimenti.
%	
%	Chiamiamo gli elementi dell'insieme $N(v) = \{ u : (v,u) \in E \}$ vicini del vertice $v$. Il numero dei vicini di $v$ è detto grado di $v$.
%	
%	Una sequenza di nodi $v_{1}, v_{2}, \ldots, v_{k}$ si dice cammino se $(v_{i}, v_{i+1}) \in E$ $\forall i = 1, \ldots k-1$; un cammino si dice semplice se $v_{i} \neq v_{j}$ $\forall i,j$ $1 \leq i \le j \leq k$.
%
%\begin{definizione}\label{def:sottografo}
%	Un grafo $G' = (V', E')$ si dice sottografo di $G = (V, E)$ se $V' \subset V$ e $E' \subset E$.
%\end{definizione}
%	
%	Scrivendo $G' \subset G$ indichiamo che $G'$ è sottografo di $G$.
%	
%\begin{definizione}\label{def:alfabeto}
%	Si dice alfabeto un insieme finito di elementi, chiamati simboli o caratteri.
%\end{definizione}
%	
%	Denotato con $\Sigma$ l'alfabeto, chiamiamo $\Sigma^{k}$ l'insieme di tutte le stringhe lunghe $k$ formate da simboli di $\Sigma$.
%	
%\begin{definizione}\label{def:grafoetichettato}
%	Si dice grafo etichettato una terna ordinata $(V,E,L)$ con $(V,E)$ grafo e $L : V \rightarrow \Sigma$ una funzione che associa ad ogni vertice $v$ un carattere (detto etichetta del vertice) dell'alfabeto $\Sigma$.   
%\end{definizione}
%
%	Dato un cammino $\pi = v_{1}, v_{2}, \ldots, v_{k}$,
%	estendiamo la funzione $L$ e indichiamo con $L(\pi) = L(v_{1}) \cdot L(v_{2}) \cdot \ldots \cdot L(v_{k})$ la stringa ottenuta concatenando\footnote{Con il simbolo $\cdot$ indichiamo la concatenazione tra due stringhe} le etichette dei singoli nodi del cammino.
%	
%
%\section{Il problema}
%
%\section{Applicazioni reali}
%
%