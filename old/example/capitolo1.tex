%!TEX TS-program = pdflatex
%!TEX root = tesi.tex
%!TEX encoding = UTF-8 Unicode


     %%%%%%%%%%%%%%%%%%%%
     %                  %
     %  capitolo1.tex   %
     %                  %
     %%%%%%%%%%%%%%%%%%%%

\chapter{Rudimenti di \TeX}

Raccogliamo qui alcune delle cose di base per un
principiante di \TeX.

\section{Spazi}

Uno o più spazi fra due parole del testo sorgente non
fanno differenza: \verb!uno due! ha lo stesso effetto
di \verb!uno     due!. Per terminare il
paragrafo e avere il rientro nella riga seguente il modo
consigliato è di lasciare una o più righe bianche dopo
la fine del paragrafo. Un'alternativa è di scrivere
\verb!\par! alla fine del paragrafo.

Un'altra equazione numerata:
\begin{equation}\label{prodottoNotevole2}
  (a-b)(a^2+ab+b^2)=a^3-b^3.
\end{equation}

La spaziatura verticale è gestita in gran parte
automaticamente dal \LaTeX. Se si vuole uno stacco è
spesso meglio vedere se non sia il caso di introdurre una
nuova \verb!\section!, \verb!\subsection!
\verb!\subsubsection!, \verb!itemize! o un enunciato
(\verb!osservazione!, \verb!corollario! ecc.), e fidarsi
della spaziatura automatica.

Se proprio si vuole inserire uno spazio manualmente si
può usare il comando
\verb!\hspace{!\textit{lunghezza}\verb!}! per gli spazi
orizzontali e \verb!\vspace{!\textit{lunghezza}\verb!}!
per gli spazi verticali. Per esempio scrivendo
\begin{center}
  \verb!\hspace{1cm}!
\end{center}
si lascia uno spazio orizzontale di 1 centimetro.
Per gli spazi verticali è meglio scegliere fra i
seguenti tre spazi standardizzati piccolo, medio e grande:
\begin{center}
  \verb!\smallskip!, \verb!\medskip!, \verb!\bigskip!.
\end{center}

\section{Trattini}

Il \TeX{} distingue quattro tipi di trattini
orizzontali. Notate come sono diversi per lunghezza,
spessore e spaziatura.

\begin{itemize}

\item il trattino semplice (esempio:
video-proiettore), che si ottiene battendo il tasto del
trattino (o~segno meno): \verb!video-proiettore!.

\item il trattino medio, che per quanto ne so si usa
solo fra due numeri per indicare l'intervallo fra i due,
per esempio nel citare le pagine da 15 a 21 di un libro
potrò scrivere ``p.~15--21''. Questo trattino si
ottiene battendo due volte il tasto del trattino semplice:
\verb!p. 15--21!.

\item il trattino lungo, poco usato in italiano, ma che
in inglese è spesso usato per delimitare un
inciso---come noi useremmo una parentesi. Si ottiene
battendo tre volte il trattino semplice:
\verb!inciso---come!. Nella tipografia inglese non si
lasciano spazi attorno al trattino lungo.

\item il segno meno, che si usa nelle formule
matematiche: $a-1$. Si ottiene battendo il trattino
semplice, ma all'interno di una formula: \verb!$a-1$!.
Qui i dollari servono per delimitare la formula.

\end{itemize}

Un'altra equazione numerata:
\begin{equation}
  (a-b)(a^3+a^2b+ab^2+b^3)=a^4-b^4.
\end{equation}
Per come è settato questo esempio di tesi, il numero di equazione è una coppia il cui primo elemento è il numero di capitolo e il secondo è il numero di formula all'interno del capitolo. Per esempio, nella formula~\eqref{prodottoNotevole2} di pag.~\pageref{prodottoNotevole2} il primo~`1' si riferisce al capitolo (non alla sezione!) e il secondo~`1' alla formula. Fanno eccezione le formule dell'introduzione, il cui numero non è una coppia ma un singolo numero d'ordine (vedi la formula~\eqref{prodottoNotevole} di pag.~\pageref{prodottoNotevole}).


\section{Accenti, lettere insolite, e cambi di font}

Alcuni esempi di lettere o accenti stranieri:
\begin{center}
\begin{tabular}{ll}
  Weierstraß& \verb!Weierstraß! \\
  Gödel & \verb!Gödel! \\
  naïve & \verb!naïve! \\
   Øystein & \verb!Øystein! \\
  Ångström & \verb!Ångström! \\
  Stanis{\l}aw \'Swierczkowski & 
    \verb!Stanis{\l}aw \'Swierczkowski! \\
  Pál Erd\H{o}s &
    \verb!Pál Erd\H{o}s! \\
  T\=oky\=o & \verb!T\=oky\=o! \\
  L'Hôpital & \verb!L'Hôpital!
\end{tabular}
\end{center}
Se ve ne servono degli altri consultate un manuale.

\medskip

Varie \emph{forme} dei caratteri e i comandi per
ottenerle:

\medskip

\begin{center}
\begin{tabular}{ll}
  \textup{Testo normale (upright)} &
    \verb!\textup{Testo normale!\\
    &\qquad\verb!(upright)}!\\ 
  \textit{corsivo (``italic'')} &
    \verb!\textit{corsivo!\\
    &\qquad\verb!(``italic'')}!\\
  \textsl{inclinato (``slanted'')} &
    \verb!\textsl{inclinato!\\
    &\qquad\verb!(``slanted'')}!\\
  \textsc{maiuscolette (``small caps'')} &
    \verb!\textsc{maiuscolette!\\
    &\qquad\verb!(``small caps'')}!
\end{tabular}
\end{center}
%
Varie \emph{serie} (o pesantezza) di caratteri:
%
\begin{center}
\begin{tabular}{ll}
  \textmd{serie media (normale)} &
    \verb!\textmd{serie media!\\
    &\qquad\verb!(normale)}!\\
  \textbf{serie grassetta (``boldface'')} &
    \verb!\textbf{grassetto!\\
    &\qquad\verb!(``boldface'')}!
\end{tabular}
\end{center}
%
Varie \emph{famiglie} di caratteri:
%
\begin{center}
\begin{tabular}{ll}
  \textrm{famiglia romana (normale)} &
    \verb!\textrm{famiglia romana!\\
    &\qquad\verb!(normale)}!\\
  \textsf{senza grazie (``sans serif'')} &
    \verb!\textsf{senza grazie!\\
    &\qquad\verb!(``sans serif'')}!\\
  \texttt{telescrivente (``typewriter'')}&
    \verb!\texttt{telescrivente!\\
    &\qquad\verb!(``typewriter'')}!
\end{tabular}
\end{center}
%
Forme, serie e famiglie si possono combinare: battendo
\begin{center}
  \verb!\textsf{\textbf{Cosa?}}!
\end{center}
si ottiene \textsf{\textbf{Cosa?}} Non tutte le
combinazioni però sono supportate.

Attenzione! Una tesi di laurea non è il luogo per
sbizzarrirsi con l'uso di caratteri strani. La parola
d'ordine è \emph{sobrietà} e da evitare sono le
\emph{pacchianate}. Normalmente non avrete bisogno di
altro che di evidenziare qualche parola qua e là nel
testo, e per questo si raccomanda di usare il comando
\verb!\emph!. Per esempio per produrre
\emph{testo enfatico} basta battere
\verb!\emph{testo enfatico}! e lasciare il resto al
\LaTeX.

Chi deve inserire nel testo dei listati di programmi di
calcolatore, si veda i manuali riguardo i comandi
\verb!\verb!, \verb!\verbatim! e i pacchetti
\verb!verbatim!, \verb!moreverb! e simili.

Il pacchetto \verb!hyperref!,\index{ipertesto} (chiamato automaticamente da \verb!uniudtesi!, agevola i riferimenti a internet:

\begin{itemize}

\item gestisce automaticamente i font adatti, e 

\item rende ``cliccabili'' i riferimenti nelle versioni su schermo (dvi o pdf).

\end{itemize}

\noindent Per esempio, scrivendo \verb!\url{http://www.dimi.uniud.it}! viene

\begin{center}
\url{http://www.dimi.uniud.it}
\end{center}