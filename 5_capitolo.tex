\chapter{Conclusion and future works}
    

We presented randomized algorithms and data structures for sketching subgraph similarity, which take into account both the internal structure of subgraphs and their interface to the rest of the network. 
The sketches are relatively small in size (near logarithmic)
and exploit the distributions of the q-grams involved. The proposed
algorithms, f-samp and f-count, guarantee a good approximation
(as unbiased estimators) of the Bray-Curtis index and the Frequency
Jaccard index, and show good practical performance compared to a
less refined baseline sampler. In particular the steady running time
of f-samp on networks with hundreds of millions of edges suggests
its usefulness as an estimator on very large networks.
The assumption that the graph is undirected with one label per
node can be removed, and it would be interesting to study further
similarity indexes that can be sketched with our algorithms.