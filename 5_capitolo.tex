\chapter{Conclusion and future works}
    

We presented randomized algorithms and data structures for sketching subgraph similarity, 
which take into account both the internal structure of subgraphs and their interface to the rest of the network. \\

The proposed algorithms, $\textsc{f-samp}$ and $\textsc{f-count}$, guarantee a good efficiency and approximation
(as unbiased estimators) of the Bray-Curtis index and the Frequency
Jaccard index, and show good practical performance compared to a
less refined baseline sampler $\textsc{base}$. In particular the steady running time
of $\textsc{f-samp}$ on networks with hundreds of millions of edges suggests
its usefulness as an estimator on very large networks.\\

A great advantage of the proposed algorithms is that they are highly parallelizable,
which makes them suitable for analyzing massive dataset using today's datacenter with 
thousands of cpu cores running simultaneously.\\

As future work, the assumption that the graph is undirected with one label per
node can be removed, and it would be interesting to study further
similarity indexes that can be sketched with our algorithms.