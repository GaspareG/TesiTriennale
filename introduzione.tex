%!TEX TS-program = pdflatex
%!TEX root = tesi.tex
%!TEX encoding = UTF-8 Unicode


       %%%%%%%%%%%%%%%%%%%%%%
       %                    %
       %  Introduzione.tex  %
       %                    %
       %%%%%%%%%%%%%%%%%%%%%%

\chapter{Introduzione}

\section{Definizioni di base e notazione}

\begin{definizione}\label{def:grafo}
	Si dice grafo una coppia ordinata $(V, E)$, con $V$ insieme dei vertici e $E \subseteq V \times V$ insieme degli archi.
\end{definizione}
	
	Se due vertici $u, v \in V$ sono congiunti da un arco si dicono estremi dell'arco, in questo caso indichiamo l'arco con la coppia $(u, v) \in E$.
	
	Se $(u,v) \in E \Leftrightarrow (v,u) \in E$ il grafo si dice non orientato. Considereremo solo grafo non orientati se non specificato altrimenti.
	
	Chiamiamo gli elementi dell'insieme $N(v) = \{ u : (v,u) \in E \}$ vicini del vertice $v$. Il numero dei vicini di $v$ è detto grado di $v$.
	
	Una sequenza di nodi $v_{1}, v_{2}, \ldots, v_{k}$ si dice cammino se $(v_{i}, v_{i+1}) \in E$ $\forall i = 1, \ldots k-1$; un cammino si dice semplice se $v_{i} \neq v_{j}$ $\forall i,j$ $1 \leq i \le j \leq k$.

\begin{definizione}\label{def:sottografo}
	Un grafo $G' = (V', E')$ si dice sottografo di $G = (V, E)$ se $V' \subset V$ e $E' \subset E$.
\end{definizione}
	
	Scrivendo $G' \subset G$ indichiamo che $G'$ è sottografo di $G$.
	
\begin{definizione}\label{def:alfabeto}
	Si dice alfabeto un insieme finito di elementi, chiamati simboli o caratteri.
\end{definizione}
	
	Denotato con $\Sigma$ l'alfabeto, chiamiamo $\Sigma^{k}$ l'insieme di tutte le stringhe lunghe $k$ formate da simboli di $\Sigma$.
	
\begin{definizione}\label{def:grafoetichettato}
	Si dice grafo etichettato una terna ordinata $(V,E,L)$ con $(V,E)$ grafo e $L : V \rightarrow \Sigma$ una funzione che associa ad ogni vertice $v$ un carattere (o etichetta) dell'alfabeto $\Sigma$.   
\end{definizione}

	Dato un cammino $\pi = v_{1}, v_{2}, \ldots, v_{k}$,
	estendiamo la funzione $L$ e indichiamo con $L(\pi) = L(v_{2}) \cdot L(v_{1}) \cdot \ldots \cdot L(v_{k})$ la stringa ottenuta concatenando le etichette dei singoli nodi del cammino.
	

\section{Il problema}

\section{Applicazioni reali}

