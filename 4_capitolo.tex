%%%%%%%%%%%%%%%%%%%%%%

\chapter{Project development}
    
    \section{Implementation choices}
    
    
    
    \section{Dataset}
    
    For the experiments we use two different kind of dataset, a small one so we can easily brute-force the real indices and compare the relative errors, and a big one in order to benchmark the performance of the different approach.
    
    \paragraph*{NetInf} This graph represents the flow of information on the web among blogs and news websites. The graph was computed by the \textit{NetInf} approach, as part of the \textit{SNAP} project~\cite{netinf}, by tracking cascades of information diffusion to reconstruct ``who copies who'' relationships.
    
    \begin{itemize}
    	\item $V$ is the set of blog or news website, $|V| = 854$.
    	\item $E$, each website is connected to those who frequently copy their content, $|E| = 3824$.
    	\item $\Sigma$ is the set of ranking class of websites ($0$ top $4\%$, $1$ next $15\%$, $2$ next $30\%$, $3$ last $51\%$), $|\Sigma| = 4$.
    	\item $L$, each website is labeled according to its importance, using Amazon's Alexa ranking.
    \end{itemize}

    % Each node represents a blog or news website, and a website is connected to those who frequently copy their content. The graph contains 854 nodes and 3824 edges. We labelled websites according to their importance, using Amazon's Alexa ranking~\cite{alexarank}: the labels correspond to respectively the websites ranked in the top $4\%$, the following $15\%$, the following $30\%$, and the remaining $51\%$ (i.e. $|\Sigma|=4$). 
    
    \textsl{Considered query:} compute the similarity of two websites $a$ and $b$ or two sets of websites.
    
    \paragraph*{IMDb} In this graph, taken from the \textit{Internet Movie Database} we have:
    
    \begin{itemize}
    	\item $V$ is the set of all movies in \textit{IMDb},  $|V| = 1\,060\,209$.
		\item $E$, two movies are connected if their casts share at least one actor, $|E| = 288\,008\,472$.
		\item $\Sigma$ is the set of movies genre, $|\Sigma| = 36$.
		\item $L$, each movie is labeled with its principal genre.
    \end{itemize}
    
    % , nodes correspond to movies, and there is a link between two movies if their casts share at least one actor. The graph contains 1\,060\,209 movies (nodes) and 288\,008\,472 edges. Each movie is labeled with one of $|\Sigma|=36$ genres. 
    
    \textsl{Considered query:} similarity of actors' ego-networks. Given two actors $a$ and $b$, let $A$ and $B$ be their ego-networks, i.e., the sets of nodes corresponding to movies in which respectively $a$ and $b$ starred, compute the similarity of $A$ and $B$.
    %Compute $BC(A,B)$, $J(A,B)$, $FJ(A,B)$, and random $q$-walks.
    
    \section{Experimental results}

    We describe the experimental evaluation for our approach. Our computing platform is a machine with Intel(R) Xeon(R) CPU E5-2620 v3 at 2.40GHz, 24 virtual cores, 128 Gb RAM, running Ubuntu Linux v.4.4.0-22-generic. Code written in C++17, compiled with g++ v.5.4.1 and OpenMP 4.5.\\
    
    To better analyze the different approaches described, we take several kinds of experiment in each of them .\footnote{Unless otherwise stated all the results are the average of $100$ identical experiment, in order to reduce the possible errors randomly caused by the machine.}\\
     
    An important fact of which to take into account is that we make large use of parallelization, 
    so all the running times scale (approximately) linearly on the number of cores used.
    
	\subsection*{Running time}
	
	In this experiment we compare the different running time, of all the parts, from all algorithms.
	
	Note that this is an important experiments, as in the real application time is crucial factor.\\

	\begin{table}[h]
		\centering
		\label{my-label}
		\begin{tabular}{|c|c|c|c|}
			\hline
			Dataset 		& $q$ & \textsc{Color-Coding} 	& \textsc{BruteForce} \\ \hline
			\textsc{NetInf}	& $3$ & $1234$					& $1234$ \\ \hline
			\textsc{NetInf}	& $3$ & $1234$					& $1234$ \\ \hline
			\textsc{NetInf}	& $3$ & $1234$					& $1234$ \\ \hline
			\textsc{NetInf}	& $3$ & $1234$					& $1234$ \\ \hline
			\textsc{IMDb}	& $3$ & $1234$					& $1234$ \\ \hline
			\textsc{IMDb}	& $4$ & $1234$					& $1234$ \\ \hline
			\textsc{IMDb}	& $5$ & $1234$					& $1234$ \\ \hline
			\textsc{IMDb}	& $6$ & $1234$					& $1234$ \\ \hline
		\end{tabular}
		\caption{Time in milliseconds}
	\end{table}
	
	\subsection*{Relative error and variance}
	
	In this experiment we will compare, for increasing value of $R$, how accurate are the different algorithms.\\
		
	TODO
	\begin{table}[h]
		\centering
		\label{my-label}
		\begin{tabular}{|c|c|c|c|}
			\hline
			Dataset 		& $q$ & \textsc{Color-Coding} 	& \textsc{BruteForce} \\ \hline
			\textsc{NetInf}	& $3$ & $1234$					& $1234$ \\ \hline
			\textsc{NetInf}	& $3$ & $1234$					& $1234$ \\ \hline
			\textsc{NetInf}	& $3$ & $1234$					& $1234$ \\ \hline
			\textsc{NetInf}	& $3$ & $1234$					& $1234$ \\ \hline
			\textsc{IMDb}	& $3$ & $1234$					& $1234$ \\ \hline
			\textsc{IMDb}	& $4$ & $1234$					& $1234$ \\ \hline
			\textsc{IMDb}	& $5$ & $1234$					& $1234$ \\ \hline
			\textsc{IMDb}	& $6$ & $1234$					& $1234$ \\ \hline
		\end{tabular}
		\caption{Time in milliseconds}
	\end{table}
		
	\subsection*{Fixed relative error}
	
	In this experiment we set the relative error $\epsilon_{r}$ and compare how many paths $R$ we need to reach such relative error.\\
	
	TODO
	\begin{table}[h]
		\centering
		\label{my-label}
		\begin{tabular}{|c|c|c|c|}
			\hline
			Dataset 		& $q$ & \textsc{Color-Coding} 	& \textsc{BruteForce} \\ \hline
			\textsc{NetInf}	& $3$ & $1234$					& $1234$ \\ \hline
			\textsc{NetInf}	& $3$ & $1234$					& $1234$ \\ \hline
			\textsc{NetInf}	& $3$ & $1234$					& $1234$ \\ \hline
			\textsc{NetInf}	& $3$ & $1234$					& $1234$ \\ \hline
			\textsc{IMDb}	& $3$ & $1234$					& $1234$ \\ \hline
			\textsc{IMDb}	& $4$ & $1234$					& $1234$ \\ \hline
			\textsc{IMDb}	& $5$ & $1234$					& $1234$ \\ \hline
			\textsc{IMDb}	& $6$ & $1234$					& $1234$ \\ \hline
		\end{tabular}
		\caption{Time in milliseconds}
	\end{table}
	
	\subsection*{Fixed time precision}
	
	In this experiment we set the computing time and compare how many path.\\
	
	TODO
	\begin{table}[h]
		\centering
		\label{my-label}
		\begin{tabular}{|c|c|c|c|}
			\hline
			Dataset 		& $q$ & \textsc{Color-Coding} 	& \textsc{BruteForce} \\ \hline
			\textsc{NetInf}	& $3$ & $1234$					& $1234$ \\ \hline
			\textsc{NetInf}	& $3$ & $1234$					& $1234$ \\ \hline
			\textsc{NetInf}	& $3$ & $1234$					& $1234$ \\ \hline
			\textsc{NetInf}	& $3$ & $1234$					& $1234$ \\ \hline
			\textsc{IMDb}	& $3$ & $1234$					& $1234$ \\ \hline
			\textsc{IMDb}	& $4$ & $1234$					& $1234$ \\ \hline
			\textsc{IMDb}	& $5$ & $1234$					& $1234$ \\ \hline
			\textsc{IMDb}	& $6$ & $1234$					& $1234$ \\ \hline
		\end{tabular}
		\caption{Time in milliseconds}
	\end{table}

	\subsection*{Actors' ego-networks}
	
	As last test, in order to show a real application easy to understand, we compare
	some pairs of actors' ego-networks (using \fcount algorithm with $q=3$ and $R=1\,000$):
	
	\begin{table}[h]
		\centering
		\begin{tabular}{c|c|l|l}
			Actor/actress & Actor/actress & BC index & FJ index\\ 
			\hline
			Stan Laurel & Oliver Hardy & 0.936167 & 0.774053 \\
			Robert De Niro & Al Pacino & 0.730935 & 0.231474\\
			Woody Allen & Meryl Streep & 0.556071 & 0.222857\\
			Meryl Streep & Roberto Benigni & 0.482909 & 0.160181\\
			%\hline
		\end{tabular}
	\end{table}

	The values respect the theory very faithfully for many reasons.
	
	The Bray-Curtis index, as we already told, is always greater than the Frequency Jaccard and takes more into account the intersection:
	the ego-networks of the famous comic duo Laurel and Hardy have a big intersection, this make the Bray-Curtis value very close to 1, however the Frequency-Jaccard is much smaller as Oliver Hardy starred in about $300$ movie without Stan Laurel.
	
	One last observation about the couple Meryl Streep and Roberto Benigni: we have a big difference between the Bray-Curtis and the Frequency Jaccard, this can be due from the fact they are both famous actors (both won the Oscar Prize) who starred with a lot of other famous actor but they haven't starred together.